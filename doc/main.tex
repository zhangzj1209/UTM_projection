\documentclass[10pt]{article}
\usepackage[english]{babel}
\usepackage[utf8x]{inputenc}
\usepackage{fullpage}
\usepackage{bm}
\usepackage{amsmath}

% set page margin
\usepackage{geometry}
\geometry{a4paper, left=1cm, right=01cm, top=2cm, bottom=2cm}

\usepackage{color}
\usepackage{hyperref}
\hypersetup{colorlinks = true, 	% set the color of the link text
	urlcolor = blue}	% set the web page link to blue

\usepackage{listings} 	% insert code

\title{\textbf{UTM Projection Analysis}}
\author{Zhengjie Zhang}
\date{May 07, 2024}

\begin{document}

\maketitle


\section{\textbf{UTM Coordinate System}}
 \textbf{UTM} (Universal Transverse Mercator Grid System) is a \textbf{Plane Rectangular Coordinate System}. This system ignores the elevation information and regards the Earth's surface as an ideal ellipsoid.


\section{\textbf{UTM Grid Zones of the World}}

\begin{itemize}
	\item \textbf{Longitude Zones:} There are 60 longitudinal projection zones numbered \textbf{1} to \textbf{60} starting at 180°W. Each of these zones is \textbf{6 degrees} wide. Region 1 covers the region from 180°W to 174°W, and then the region number increases eastward until Region 60, covering the region from 174°E to 180°E. 
	
	\item \textbf{Latitude Zones:} There are 20 latitudinal zones spanning the latitudes 80°S to 84°N and denoted by the letters \textbf{C} to \textbf{X}, ommitting the letters I and O. Each of these is \textbf{8 degrees} south-north, apart from zone X which is 12 degrees south-north. N is the first north latitude zone, the letters after N belong to the north latitude zones, and the letters before N belong to the south latitude zones. It is worth noting that the polar regions further south at 80°S and further north at 84°N are not included in this system.
	
	\item Specific UTM grid zones can refer to the \textbf{\href{https://www.dmap.co.uk/utmworld.htm}{link}}.
	
	\item As approaching the boundary of the UTM region, the scale distortion will gradually increase. However, in practice, we often need to measure a series of positions in two adjacent areas, so it is particularly convenient and necessary to use a single grid for measurement. If necessary, we can appropriately extend the measurement results to a certain range of adjacent areas.

\end{itemize}


\section{\textbf{WGS84 and UTM}}	

\begin{itemize}
	\item \textbf{WGS84:} A coordinate system used by the \textbf{GPS} (Global Positioning System), which uses longitude and latitude to indicate geographic location. WGS84 is a coordinate system based on the center of the Earth, that is, its origin is the centroid of the Earth.
	
	\item \textbf{UTM:} This is a system that uses a \textbf{2-D Cartesian} coordinate system to represent the geographical location. It divides the Earth's surface into multiple regions (except for the near-Arctic and near-Antarctic regions), each of which uses its own Plane Rectangular Coordinate System. UTM is a surface-based coordinate system, that is, its origin is a point on the surface of the Earth. 
	
	\item WGS84 is the spherical coordinates, including latitude and longitude, the unit is \textbf{degree}. UTM is a plane coordinate, including x- and y- coordinates, the unit is \textbf{meter}.
	
	\item \textbf{Conversion formula:}  
	  \begin{equation*}
	  	  \bm{Longitude \; Zone} = int  \left [ \frac{\bm{Longitude}}{6} \right ]  + 31
	  \end{equation*}
	
\end{itemize}	


\section{\textbf{Tools and Codes}}	

\begin{itemize}
	\item \href{https://epsg.io/map#srs=4326&x=117.290039&y=31.952162&z=6&layer=streets}{epsg.io}
	
	\item Python Package \href{https://github.com/Turbo87/utm}{UTM}
	
\end{itemize}

\end{document}
	 